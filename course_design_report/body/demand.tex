% !Mode:: "TeX:UTF-8"

\chapter{需求概述}
\section{项目背景}

当今,随着物质生活水平的提高和网络的普及,人们对于生活条件的要求也日益提高,因此基于物联网的智能家居应运而生。

而ESP8266是一款超低功耗的UART-WiFi透传模块,拥有业内极富竞争力的封装尺寸和超低能耗技术,专为移动设备和物联网应用设计,可将用户的物理设备连接到Wi-Fi 无线网络上,进行互联网或局域网通信,实现联网功能。ESP8266可广泛应用于智能电网、智能交通、智能家具、手持设备、工业控制等领域。

在此背景之下,我们小组选择利用ESP8266制作一个网络控制检测系统,简单模拟智能家居的工作原理,进而向智能家居方向发展。

\section{项目需求}

项目的实际需求是旨在利用型号为ESP8266的单片机的WiFi模组,结合各种传感器,实现对家庭中的部分电器设备的远程操控和对家庭环境状况的远程监控。

最终产品的具体功能有:远程控制LED灯的开关,方便用户对于家电的控制;实时监测室内温度湿度数值以及变化曲线,便于用户了解室内环境情况;检测室内烟雾情况以及是否出现明火并及时发出警报,提供了非常快捷灵敏的警报系统,为用户的安全提供了有力的保障;此外,使用红外检测,实现人来灯亮、人走灯灭的全自动化功能,让用户体验到更加人性化的功能。

本产品的具体实现方式为Android手机APP和Web网页两种方式,手机APP为用户提供了便捷、便携、便利的使用体验,而Web网页为用户提供了双重保障,从而进一步提升用户体验。

\section{条件限制}

\begin{enumerate}
    \item 由于实际电路中电压太高,存在一定的安全风险,而且电路连接不方便,所以用LED灯代替电灯。
    \item 由于一台ESP8266只有一个ADC口,不能同时接多个传感器,所以用两个单片机同时工作。
    \item 由于远程网络通信更加复杂,所以暂时实现局域网内数据通信。
\end{enumerate}